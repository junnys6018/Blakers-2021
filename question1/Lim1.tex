\documentclass[11pt]{article}
\usepackage[a4paper, margin=1in]{geometry}
\usepackage{amsfonts,amsmath,amssymb,mathtools}
\usepackage[none]{hyphenat}
\usepackage{fancyhdr}
\usepackage{graphicx}
\usepackage[parfill]{parskip} %Removes indentation on new paragraph 

\pagestyle{fancy}
\fancyhead{} %clears default header
\fancyfoot{} %clears default footer
\fancyhead[L]{Ethan Lim} %\slshape makes itallics

\begin{document}

\section*{Question 1}
We first prove a more simple inequality, $\forall a,b\in\mathbb{R}:\: a^2-ab+b^2\geq0$,
with equality iff $a=b=0$. If $a$ and $b$ have different signs, then $-ab\geq0$,
and so $a^2-ab+b^2\geq0$. So suppose $a$ and $b$ have the same sign, note the following,
\begin{align}
	(a-b)^2 & \geq0 \nonumber \\
	a^2-2ab+b^2 & \geq0 \nonumber \\
	a^2-ab+b^2 & \geq ab 
\end{align}

If $a$ and $b$ have the same sign, then $ab\geq0$, so (1) implies that $a^2-ab+b^2\geq0$.

We now show that $a^2-ab+b^2=0$ iff $a=b=0$. The reverse direction is clear,
so suppose that $a^2-ab+b^2=0$. Now it cannot be the case that $a>0$ and $b<0$, since
if that were the case, then $-ab>0$, implying that $a^2-ab+b^2>0$. By symmetric argument,
it cannot be that $a<0$ and $b>0$. Hence either $a,b\geq0$ or $a,b\leq0$, then it follows that
$ab\geq0$, also (1) implies that $ab\leq a^2-ab+b^2=0$. So it must be that $ab=0$, hence
$a=0$ or $b=0$, without loss of generality, assume $a=0$, then $a^2-ab+b^2=0$ implies that
$b^2=0$ so $b=0$. 

With this inequality established, let $a=x-1$ and $b=y-1$, then,
\begin{align*}
	(x-1)^2-(x-1)(y-1)+(y-1)^2 & \geq 0 \\
	(x^2-2x+1)-(xy-x-y+1)+(y^2-2y+1) & \geq 0 \\
	x^2+y^2-xy-x-y+1 & \geq 0 \\ 
	x^2+y^2+2-(xy+x+y+1) & \geq 0 \\ 
	x^2+y^2+2-(x+1)(y+1) & \geq 0
\end{align*}
Which implies that $x^2+y^2+2\geq(x+1)(y+1)$, as required. We have equality iff
$a=x-1=0$ and $b=y-1=0$. ie iff $x=y=1$
\end{document}
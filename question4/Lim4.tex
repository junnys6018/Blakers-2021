\documentclass[11pt]{article}
\usepackage[a4paper, margin=1in]{geometry}
\usepackage{amsfonts,amsmath,amssymb,mathtools}
\usepackage[none]{hyphenat}
\usepackage{fancyhdr}
\usepackage{graphicx}
\usepackage[parfill]{parskip} %Removes indentation on new paragraph 

\pagestyle{fancy}
\fancyhead{} %clears default header
\fancyfoot{} %clears default footer
\fancyhead[L]{Ethan Lim} %\slshape makes itallics
\fancyhead[R]{\thepage}

\begin{document}

\section*{Question 4}
(Note: I use the convention that $\mathbb{N}$ does not include $0$)

We will use numbers and sequences to represent letters and words, this is to make notation easier to work with. So for example, `a' will be represented by $0$, `b' with $1$, and `z' with $25$. The word \textit{abdc} will be represented by the sequence $(0,1,3,2)$.

We define a factorial number system, where the $i$th digit has base $i$, and the place value of the $i$th digit is $(i-1)!$. A number in the factorial system is written with pipes separating the digits and subscripted with `!'. For example, $3|0|2|1|0_!=3\times4!+0\times3!+2\times2!+1\times1!+0\times0!=77$. The highest digit allowed in the $i$th digit is $i-1$, hence the first digit in the factorial number system is always zero.

To convert a number from base 10 to the factorial system we may use repeated division, for example to convert $463$
\begin{itemize}
	\item 463 ÷ 1 = 463, remainder 0
	\item 463 ÷ 2 = 231, remainder 1
	\item 231 ÷ 3 = 77, remainder 0
	\item 77 ÷ 4 = 19, remainder 1
	\item 19 ÷ 5 = 3, remainder 4
	\item 3 ÷ 6 = 0, remainder 3
\end{itemize}
So $463=3|4|1|0|1|0_!$. 

Fix some $n\in\mathbb{N}$. A sequence $(a_0,...,a_{n-1})$ of natural numbers is a \textit{good} sequence just in case each number $0, 1,...,n-1$ appears exactly once in the sequence. Let $A$ be the set of good sequences. Note that $|A|=n!$. We may define a total order on $A$ by way of lexicographic order, it is easy to see that the words will be published in lexicographic order. Let $B=\{0,1,2,...,n!-1\}$ we define a map $f:B\to A$ given by 
\[f(x_0|x_1|...|{x_{n-1}}_!)=(L(S_0)_{x_0},L(S_1)_{x_1},...,L(S_{n-1})_{x_{n-1}})\]
 Where $L(S)$ takes a finite set of natural numbers and returns the sequence of natural numbers in $S$ in strictly increasing order, $L(S)_i$ denotes the $i$th element in the sequence (starting from zero), and $S_0=\{0,...,n-1\}$,$S_1=S_0\setminus\{L(S_0)_{x_0}\}$, $S_2=S_1\setminus\{L(S_1)_{x_1}\}$ etc...

An example will clear things up, take $n=4$ and consider $f(3)=(a_0,a_1,a_2,a_3)$. Note that
$3=0|1|1|0_!$.
\begin{itemize}
	\item $S_0=\{0,1,2,3\}$, hence $L(S_0)=(0,1,2,3)$, so $a_0=L(S_0)_0=(0,1,2,3)_0=0$.
	\item $S_1=S_0\setminus{a_0}=\{1,2,3\}$, hence, $L(S_1)=(1,2,3)$, so $a_1=L(S_1)_1=(1,2,3)_1=2$.
	\item $S_2=S_1\setminus{a_1}=\{1,3\}$, hence $L(S_2)=(1,3)$, so $a_2=L(S_2)_1=(1,3)_1=3$.
	\item $S_3=S_2\setminus{a_2}=\{1\}$, hence $L(S_3)=(3)$, so $a_3=L(S_3)_0=(1)_0=1$.
\end{itemize}
Hence $f(3)=(0,2,3,1)$

It remains to show that for every $x_0|x_1|...|{x_{n-1}}_!\in B$, $f(x_0|x_1|...|{x_{n-1}}_!)=(a_0,...,a_n)$ does give a good sequence, and indexing the sequence $L(S_i)$ with index $x_i$ does not go out of bounds. This is the case since at each $i$, $S_i=S_0\setminus\{a_0,...,a_{i-1}\}$, hence $a_i$ cannot be equal to any of $a_0,...,a_{i-1}$. The indexing does not go out of bounds since $x_i$ is at the $n-i$th digit and by definition of the factorial number system, $x_i<n-i$, in addition, $|S_i|=n-i$, so $L(S_i)_{x_i}$ is well defined.

Claim: $f$ is strictly increasing

Let $x=x_0|x_1|...|{x_{n-1}}_!$ and $y=y_0|y_1|...|{y_{n-1}}_!$ such that $x<y$. Then there exist an $i$ such that $x_0=y_0$, $x_1=y_1$,...,$x_{i-1}=y_{i-1}$ and $x_i<y_i$. Let $f(x)=(a_0,...,a_{n-1})$ and $f(y)=(b_0,...,b_{n-1})$ then it follows that $a_0=b_0$, $a_1=b_1$,...,$a_{i-1}=b_{i-1}$ (since the last $i$ digits of $x$ and $y$ are the same). Hence $S_i=S_0\setminus\{a_0,...,a_{i-1}\}=S_0\setminus\{b_0,...,b_{i-1}\}$, now $a_i=L(S_i)_{x_i}$ and $b_i=L(S_i)_{y_i}$, since $L(S_i)$ is and increasing sequence, and $x_i<y_i$, it follows that $a_i<b_i$. And so $(a_0,...,a_{n-1})<(b_0,...,b_{n-1})$ (lexicographically)

Since $f$ is a strictly increasing function between two finite totally ordered sets of the same size, it follows that $f$ is a bijection. Taking $n=26$ it follows that $f(0)$ corresponds to the first word in the list, $f(1)$, the second, etc...

With the relevant theory established, we return to the question at hand. There are $26!/21$ words in a volume, we wish to find the last word of the first volume, i.e. the $26!/21$th word, which corresponds to $f(26!/21-1)$. We will calculate $26!/21$ in the factorial number system then subtract 1. Since $20!$ divides $26!/21$, the first $20$ digits of $26!/21$ are $0$. The quotient after $20$ divisions is $26\times25\times24\times23\times22=7893600$. Here are the rest of the steps

\begin{itemize}
	\item 7893600 ÷ 21 = 375885, remainder 15
	\item 375885 ÷ 22 = 17085, remainder 15
	\item 17085 ÷ 23 = 742, remainder 19
	\item 742 ÷ 24 = 30, remainder 22
	\item 30 ÷ 25 = 1, remainder 5
	\item 1 ÷ 26 = 0, remainder 1
\end{itemize}
Hence, $26!/21=1|5|22|19|15|15|0|0|0|0|0|0|0|0|0|0|0|0|0|0|0|0|0|0|0|0_!$,\\
so $26!/21-1=1|5|22|19|15|14|19|18|17|16|15|14|13|12|11|10|9|8|7|6|5|4|3|2|1|0_!$.
Hence
\begin{align*}
	f(26!/21-1)&=(1,6,24,21,17,16,25,23,22,20,19,18,15,14,13,12,11,10,9,8,7,5,4,3,2,0) \\
	&=\text{\textit{bgyvrqzxwutsponmlkjihfedca}}
\end{align*} 

\end{document}
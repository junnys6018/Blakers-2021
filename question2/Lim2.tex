\documentclass[11pt]{article}
\usepackage[a4paper, margin=1in]{geometry}
\usepackage{amsfonts,amsmath,amssymb,mathtools}
\usepackage[none]{hyphenat}
\usepackage{fancyhdr}
\usepackage{graphicx}
\usepackage[parfill]{parskip} %Removes indentation on new paragraph 

\pagestyle{fancy}
\fancyhead{} %clears default header
\fancyfoot{} %clears default footer
\fancyhead[L]{Ethan Lim} %\slshape makes itallics
\fancyhead[R]{\thepage}

\begin{document}

\section*{Question 2}
Note that $ab+bc+cd+da=(b+d)(a+c)$ so our equation becomes $(b+d)(a+c)=2021$, since $a,b,c,d\in\mathbb{N}$, it follows that $b+d,a+c\in\mathbb{N}$. Now 2021 can be expressed as a product of two positive integers in only 2 ways, $1\cdot2021$ and $43\cdot47$. Now since $b,d\geq 1$, it follows that $b+d\geq 2$, so it cannot be that $b+d=1$ and $a+c=2021$. Similarly it cant be that $b+d=2021$ and $a+c=1$. Hence the solution to our equation are those $(a,b,c,d)$ such that $b+d=43$ and $a+c=47$ or $b+d=47$ and $a+c=43$. There are 42 ways in which $b+d=43$, namely $(b,d)=(1,42),(2,41),(3,40),...,(41,2),(42,1)$, similarly there are 46 ways in which $a+c=47$, hence the total number of solutions is $42\cdot46\cdot2=3864$
\end{document}
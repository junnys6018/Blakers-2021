\documentclass[11pt]{article}
\usepackage[a4paper, margin=1in]{geometry}
\usepackage{amsfonts,amsmath,amssymb,mathtools}
\usepackage[none]{hyphenat}
\usepackage{fancyhdr}
\usepackage{graphicx}
\usepackage[parfill]{parskip} %Removes indentation on new paragraph 

\pagestyle{fancy}
\fancyhead{} %clears default header
\fancyfoot{} %clears default footer
\fancyhead[L]{Ethan Lim} %\slshape makes itallics
\fancyhead[R]{\thepage}

\begin{document}

\section*{Question 3}
\begin{align}
	x+\frac{1}{x}&=2y^2 \\
	y+\frac{1}{y}&=2z^2 \\
	z+\frac{1}{z}&=2x^2
\end{align}
Let $(x,y,z)$ be a solution to the system of equations. Note that $x,y,z>0$, since if $x<0$ then $x+\frac{1}{x}<0$ and there would be no solution to equation (1) (since $2y^2$ is always non negative). By symmetric argument it follows that $y,z>0$.

We can improve this bound to $x,y,z\geq 1$
\begin{align*}
	(x-1)^2&\geq 0 \\
	x^2-2x+1&\geq 0 \\
	x^2+1&\geq 2x \\
	x + \frac{1}{x} &\geq 2 && \text{The direction of the inequality is preserved since $x>0$}
\end{align*}

By (1) it follows that $2y^2\geq 2$, which along with the fact that $y>0$, implies that $y\geq 1$. Again, by symmetric argument, it follows that $x,z\geq 1$. I claim that $x=y=z$, note that we can rearrange (1) to give us $y=\sqrt{\frac{1}{2}\left( x+\frac{1}{x}\right)}$
\begin{align*}
	x&\leq x^2 & \text{since $x\geq 1$} \\
	2x&\leq 2x^2 \\
	x+x&\leq 2x^2 \\
	x+\frac{1}{x}&\leq 2x^2  & \text{since $\frac{1}{x}\leq x$} \\
	\frac{1}{2}\left( x+\frac{1}{x}\right) &\leq x^2 \\
	\sqrt{\frac{1}{2}\left( x+\frac{1}{x}\right)} &\leq x \\
	y&\leq x & \text{By eqn (1)}
\end{align*}
Applying the same argument to eqn (2) and (3) we see that $z\leq y$ and $x\leq z$, which implies that $x=y=z$. Now suppose that $x>1$, then $x<x^2$, and we can apply the same argument as above to conclude that $y<x$ which is a contradiction to $x=y$, hence it must be that $x=1$ so the only solution to the equations is $(x,y,z)=(1,1,1)$.
\end{document}